% Options for packages loaded elsewhere
\PassOptionsToPackage{unicode}{hyperref}
\PassOptionsToPackage{hyphens}{url}
%
\documentclass[
]{article}
\usepackage{amsmath,amssymb}
\usepackage{iftex}
\ifPDFTeX
  \usepackage[T1]{fontenc}
  \usepackage[utf8]{inputenc}
  \usepackage{textcomp} % provide euro and other symbols
\else % if luatex or xetex
  \usepackage{unicode-math} % this also loads fontspec
  \defaultfontfeatures{Scale=MatchLowercase}
  \defaultfontfeatures[\rmfamily]{Ligatures=TeX,Scale=1}
\fi
\usepackage{lmodern}
\ifPDFTeX\else
  % xetex/luatex font selection
\fi
% Use upquote if available, for straight quotes in verbatim environments
\IfFileExists{upquote.sty}{\usepackage{upquote}}{}
\IfFileExists{microtype.sty}{% use microtype if available
  \usepackage[]{microtype}
  \UseMicrotypeSet[protrusion]{basicmath} % disable protrusion for tt fonts
}{}
\makeatletter
\@ifundefined{KOMAClassName}{% if non-KOMA class
  \IfFileExists{parskip.sty}{%
    \usepackage{parskip}
  }{% else
    \setlength{\parindent}{0pt}
    \setlength{\parskip}{6pt plus 2pt minus 1pt}}
}{% if KOMA class
  \KOMAoptions{parskip=half}}
\makeatother
\usepackage{xcolor}
\usepackage[margin=1in]{geometry}
\usepackage{color}
\usepackage{fancyvrb}
\newcommand{\VerbBar}{|}
\newcommand{\VERB}{\Verb[commandchars=\\\{\}]}
\DefineVerbatimEnvironment{Highlighting}{Verbatim}{commandchars=\\\{\}}
% Add ',fontsize=\small' for more characters per line
\usepackage{framed}
\definecolor{shadecolor}{RGB}{248,248,248}
\newenvironment{Shaded}{\begin{snugshade}}{\end{snugshade}}
\newcommand{\AlertTok}[1]{\textcolor[rgb]{0.94,0.16,0.16}{#1}}
\newcommand{\AnnotationTok}[1]{\textcolor[rgb]{0.56,0.35,0.01}{\textbf{\textit{#1}}}}
\newcommand{\AttributeTok}[1]{\textcolor[rgb]{0.13,0.29,0.53}{#1}}
\newcommand{\BaseNTok}[1]{\textcolor[rgb]{0.00,0.00,0.81}{#1}}
\newcommand{\BuiltInTok}[1]{#1}
\newcommand{\CharTok}[1]{\textcolor[rgb]{0.31,0.60,0.02}{#1}}
\newcommand{\CommentTok}[1]{\textcolor[rgb]{0.56,0.35,0.01}{\textit{#1}}}
\newcommand{\CommentVarTok}[1]{\textcolor[rgb]{0.56,0.35,0.01}{\textbf{\textit{#1}}}}
\newcommand{\ConstantTok}[1]{\textcolor[rgb]{0.56,0.35,0.01}{#1}}
\newcommand{\ControlFlowTok}[1]{\textcolor[rgb]{0.13,0.29,0.53}{\textbf{#1}}}
\newcommand{\DataTypeTok}[1]{\textcolor[rgb]{0.13,0.29,0.53}{#1}}
\newcommand{\DecValTok}[1]{\textcolor[rgb]{0.00,0.00,0.81}{#1}}
\newcommand{\DocumentationTok}[1]{\textcolor[rgb]{0.56,0.35,0.01}{\textbf{\textit{#1}}}}
\newcommand{\ErrorTok}[1]{\textcolor[rgb]{0.64,0.00,0.00}{\textbf{#1}}}
\newcommand{\ExtensionTok}[1]{#1}
\newcommand{\FloatTok}[1]{\textcolor[rgb]{0.00,0.00,0.81}{#1}}
\newcommand{\FunctionTok}[1]{\textcolor[rgb]{0.13,0.29,0.53}{\textbf{#1}}}
\newcommand{\ImportTok}[1]{#1}
\newcommand{\InformationTok}[1]{\textcolor[rgb]{0.56,0.35,0.01}{\textbf{\textit{#1}}}}
\newcommand{\KeywordTok}[1]{\textcolor[rgb]{0.13,0.29,0.53}{\textbf{#1}}}
\newcommand{\NormalTok}[1]{#1}
\newcommand{\OperatorTok}[1]{\textcolor[rgb]{0.81,0.36,0.00}{\textbf{#1}}}
\newcommand{\OtherTok}[1]{\textcolor[rgb]{0.56,0.35,0.01}{#1}}
\newcommand{\PreprocessorTok}[1]{\textcolor[rgb]{0.56,0.35,0.01}{\textit{#1}}}
\newcommand{\RegionMarkerTok}[1]{#1}
\newcommand{\SpecialCharTok}[1]{\textcolor[rgb]{0.81,0.36,0.00}{\textbf{#1}}}
\newcommand{\SpecialStringTok}[1]{\textcolor[rgb]{0.31,0.60,0.02}{#1}}
\newcommand{\StringTok}[1]{\textcolor[rgb]{0.31,0.60,0.02}{#1}}
\newcommand{\VariableTok}[1]{\textcolor[rgb]{0.00,0.00,0.00}{#1}}
\newcommand{\VerbatimStringTok}[1]{\textcolor[rgb]{0.31,0.60,0.02}{#1}}
\newcommand{\WarningTok}[1]{\textcolor[rgb]{0.56,0.35,0.01}{\textbf{\textit{#1}}}}
\usepackage{graphicx}
\makeatletter
\def\maxwidth{\ifdim\Gin@nat@width>\linewidth\linewidth\else\Gin@nat@width\fi}
\def\maxheight{\ifdim\Gin@nat@height>\textheight\textheight\else\Gin@nat@height\fi}
\makeatother
% Scale images if necessary, so that they will not overflow the page
% margins by default, and it is still possible to overwrite the defaults
% using explicit options in \includegraphics[width, height, ...]{}
\setkeys{Gin}{width=\maxwidth,height=\maxheight,keepaspectratio}
% Set default figure placement to htbp
\makeatletter
\def\fps@figure{htbp}
\makeatother
\setlength{\emergencystretch}{3em} % prevent overfull lines
\providecommand{\tightlist}{%
  \setlength{\itemsep}{0pt}\setlength{\parskip}{0pt}}
\setcounter{secnumdepth}{-\maxdimen} % remove section numbering
\ifLuaTeX
  \usepackage{selnolig}  % disable illegal ligatures
\fi
\usepackage{bookmark}
\IfFileExists{xurl.sty}{\usepackage{xurl}}{} % add URL line breaks if available
\urlstyle{same}
\hypersetup{
  pdftitle={Individual Assignment II},
  pdfauthor={Group(1) Kyaw Min Khaing},
  hidelinks,
  pdfcreator={LaTeX via pandoc}}

\title{Individual Assignment II}
\author{Group(1) Kyaw Min Khaing}
\date{}

\begin{document}
\maketitle

\subsubsection{Load necessary packages}\label{load-necessary-packages}

\begin{Shaded}
\begin{Highlighting}[]
\FunctionTok{library}\NormalTok{(pacman)}
\FunctionTok{p\_load}\NormalTok{(tidyverse,rstatix,randtests,ggpubr)}
\end{Highlighting}
\end{Shaded}

\subsection{No-1(a) A physician who specializes in genetic diseases
develops a theory which predicts that two-thirds of the people who
develop a disease called cyclomeiosis will be males. She randomly
selects 300 people who are afflicted with cyclomeiosis and observes that
140 of them are females. Is the physician's theory supported? Above the
median (Males30 Females60) Below the median (Males70
Females60)}\label{no-1a-a-physician-who-specializes-in-genetic-diseases-develops-a-theory-which-predicts-that-two-thirds-of-the-people-who-develop-a-disease-called-cyclomeiosis-will-be-males.-she-randomly-selects-300-people-who-are-afflicted-with-cyclomeiosis-and-observes-that-140-of-them-are-females.-is-the-physicians-theory-supported-above-the-median-males30-females60-below-the-median-males70-females60}

\begin{Shaded}
\begin{Highlighting}[]
\NormalTok{observed }\OtherTok{\textless{}{-}} \FunctionTok{c}\NormalTok{(}\DecValTok{160}\NormalTok{, }\DecValTok{140}\NormalTok{)  }\CommentTok{\# Males, Females}
\NormalTok{total }\OtherTok{\textless{}{-}} \FunctionTok{sum}\NormalTok{(observed)  }\CommentTok{\# Total sample size}
\CommentTok{\# Expected counts based on the physician\textquotesingle{}s theory}
\NormalTok{expected }\OtherTok{\textless{}{-}} \FunctionTok{c}\NormalTok{(}\DecValTok{2}\SpecialCharTok{/}\DecValTok{3} \SpecialCharTok{*}\NormalTok{ total, }\DecValTok{1}\SpecialCharTok{/}\DecValTok{3} \SpecialCharTok{*}\NormalTok{ total)}
\CommentTok{\# Combine into a data frame}
\NormalTok{data }\OtherTok{\textless{}{-}} \FunctionTok{tibble}\NormalTok{(}
  \AttributeTok{Gender =} \FunctionTok{c}\NormalTok{(}\StringTok{"Males"}\NormalTok{, }\StringTok{"Females"}\NormalTok{),}
  \AttributeTok{Observed =}\NormalTok{ observed,}
  \AttributeTok{Expected =}\NormalTok{ expected}
\NormalTok{)}
\CommentTok{\# Display the data}
\NormalTok{data}
\end{Highlighting}
\end{Shaded}

\begin{verbatim}
## # A tibble: 2 x 3
##   Gender  Observed Expected
##   <chr>      <dbl>    <dbl>
## 1 Males        160      200
## 2 Females      140      100
\end{verbatim}

Performing the Chi-Square Test

\begin{Shaded}
\begin{Highlighting}[]
\CommentTok{\# Perform the Chi{-}Square Goodness{-}of{-}Fit Test}
\NormalTok{chisq\_test }\OtherTok{\textless{}{-}} \FunctionTok{chisq.test}\NormalTok{(}\AttributeTok{x =}\NormalTok{ observed, }\AttributeTok{p =} \FunctionTok{c}\NormalTok{(}\DecValTok{2}\SpecialCharTok{/}\DecValTok{3}\NormalTok{, }\DecValTok{1}\SpecialCharTok{/}\DecValTok{3}\NormalTok{))}
\CommentTok{\# Display the results}
\NormalTok{chisq\_test}
\end{Highlighting}
\end{Shaded}

\begin{verbatim}
## 
##  Chi-squared test for given probabilities
## 
## data:  observed
## X-squared = 24, df = 1, p-value = 9.634e-07
\end{verbatim}

Visualize observed vs.~expected counts

\begin{Shaded}
\begin{Highlighting}[]
\NormalTok{data }\SpecialCharTok{\%\textgreater{}\%}
  \FunctionTok{pivot\_longer}\NormalTok{(}\AttributeTok{cols =} \FunctionTok{c}\NormalTok{(Observed, Expected), }\AttributeTok{names\_to =} \StringTok{"Type"}\NormalTok{, }\AttributeTok{values\_to =} \StringTok{"Count"}\NormalTok{) }\SpecialCharTok{\%\textgreater{}\%}
  \FunctionTok{ggplot}\NormalTok{(}\FunctionTok{aes}\NormalTok{(}\AttributeTok{x =}\NormalTok{ Gender, }\AttributeTok{y =}\NormalTok{ Count, }\AttributeTok{fill =}\NormalTok{ Type)) }\SpecialCharTok{+}
  \FunctionTok{geom\_bar}\NormalTok{(}\AttributeTok{stat =} \StringTok{"identity"}\NormalTok{, }\AttributeTok{position =} \StringTok{"dodge"}\NormalTok{) }\SpecialCharTok{+}
  \FunctionTok{labs}\NormalTok{(}
    \AttributeTok{title =} \StringTok{"Observed vs. Expected Counts"}\NormalTok{,}
    \AttributeTok{x =} \StringTok{"Gender"}\NormalTok{,}
    \AttributeTok{y =} \StringTok{"Count"}\NormalTok{,}
    \AttributeTok{fill =} \StringTok{"Type"}
\NormalTok{  ) }\SpecialCharTok{+}
  \FunctionTok{theme\_minimal}\NormalTok{()}
\end{Highlighting}
\end{Shaded}

\includegraphics{Assigment2_files/figure-latex/2(a)Visualize-1.pdf}

\subsection{No-1(b) A study is conducted to determine whether five-year
old females are more likely than five-year old males to score above the
population median on a standardized test of eye-hand coordination. One
hundred randomly selected females and 100 randomly selected males are ©
2000 by Chapman \& Hall/CRC administered the test of eye-hand
coordination, and categorized with respect to whether they score above
or below the overall population median (i.e., the 50th percentile for
both males and females). Table 16.11 summarizes the results of the
study. Do the data indicate that there are gender differences in
performance? Above the median (Males30 Females60) Below the median
(Males70
Females60)}\label{no-1b-a-study-is-conducted-to-determine-whether-five-year-old-females-are-more-likely-than-five-year-old-males-to-score-above-the-population-median-on-a-standardized-test-of-eye-hand-coordination.-one-hundred-randomly-selected-females-and-100-randomly-selected-males-are-2000-by-chapman-hallcrc-administered-the-test-of-eye-hand-coordination-and-categorized-with-respect-to-whether-they-score-above-or-below-the-overall-population-median-i.e.-the-50th-percentile-for-both-males-and-females.-table-16.11-summarizes-the-results-of-the-study.-do-the-data-indicate-that-there-are-gender-differences-in-performance-above-the-median-males30-females60-below-the-median-males70-females60}

\begin{Shaded}
\begin{Highlighting}[]
\CommentTok{\# Create a dataframe for the study}
\NormalTok{study\_data }\OtherTok{\textless{}{-}} \FunctionTok{data.frame}\NormalTok{(}
  \AttributeTok{Gender =} \FunctionTok{c}\NormalTok{(}\StringTok{"Males"}\NormalTok{, }\StringTok{"Females"}\NormalTok{),}
  \AttributeTok{Above\_Median =} \FunctionTok{c}\NormalTok{(}\DecValTok{30}\NormalTok{, }\DecValTok{60}\NormalTok{),}
  \AttributeTok{Below\_Median =} \FunctionTok{c}\NormalTok{(}\DecValTok{70}\NormalTok{, }\DecValTok{40}\NormalTok{)}
\NormalTok{)}
\CommentTok{\# Print the data}
\FunctionTok{glimpse}\NormalTok{(study\_data)}
\end{Highlighting}
\end{Shaded}

\begin{verbatim}
## Rows: 2
## Columns: 3
## $ Gender       <chr> "Males", "Females"
## $ Above_Median <dbl> 30, 60
## $ Below_Median <dbl> 70, 40
\end{verbatim}

We use a Chi-Square test for independence to determine if there is a
significant association between gender and test performance.

\begin{Shaded}
\begin{Highlighting}[]
\CommentTok{\# Create a contingency table}
\NormalTok{contingency\_table }\OtherTok{\textless{}{-}} \FunctionTok{as.table}\NormalTok{(}\FunctionTok{rbind}\NormalTok{(}
  \FunctionTok{c}\NormalTok{(}\DecValTok{30}\NormalTok{, }\DecValTok{70}\NormalTok{),  }\CommentTok{\# Males}
  \FunctionTok{c}\NormalTok{(}\DecValTok{60}\NormalTok{, }\DecValTok{40}\NormalTok{)   }\CommentTok{\# Females}
\NormalTok{))}
\FunctionTok{rownames}\NormalTok{(contingency\_table) }\OtherTok{\textless{}{-}} \FunctionTok{c}\NormalTok{(}\StringTok{"Males"}\NormalTok{, }\StringTok{"Females"}\NormalTok{)}
\FunctionTok{colnames}\NormalTok{(contingency\_table) }\OtherTok{\textless{}{-}} \FunctionTok{c}\NormalTok{(}\StringTok{"Above Median"}\NormalTok{, }\StringTok{"Below Median"}\NormalTok{)}
\CommentTok{\# Perform the Chi{-}Square test}
\NormalTok{chisq\_test }\OtherTok{\textless{}{-}} \FunctionTok{chisq.test}\NormalTok{(contingency\_table)}
\CommentTok{\# Display the results}
\NormalTok{chisq\_test}
\end{Highlighting}
\end{Shaded}

\begin{verbatim}
## 
##  Pearson's Chi-squared test with Yates' continuity correction
## 
## data:  contingency_table
## X-squared = 16.99, df = 1, p-value = 3.758e-05
\end{verbatim}

\subsubsection{Report significance}\label{report-significance}

\begin{Shaded}
\begin{Highlighting}[]
\NormalTok{alpha }\OtherTok{\textless{}{-}} \FloatTok{0.05}  \CommentTok{\# Significance level}
\ControlFlowTok{if}\NormalTok{ (chisq\_test}\SpecialCharTok{$}\NormalTok{p.value }\SpecialCharTok{\textless{}}\NormalTok{ alpha) \{}
  \FunctionTok{cat}\NormalTok{(}\StringTok{"The result is significant at α ="}\NormalTok{, alpha, }
      \StringTok{"}\SpecialCharTok{\textbackslash{}n}\StringTok{(p{-}value ="}\NormalTok{, }\FunctionTok{round}\NormalTok{(chisq\_test}\SpecialCharTok{$}\NormalTok{p.value, }\DecValTok{4}\NormalTok{), }\StringTok{")."}\NormalTok{,}
      \StringTok{"}\SpecialCharTok{\textbackslash{}n}\StringTok{There is evidence of a relationship between gender and performance.}\SpecialCharTok{\textbackslash{}n}\StringTok{"}\NormalTok{)}
\NormalTok{\} }\ControlFlowTok{else}\NormalTok{ \{}
  \FunctionTok{cat}\NormalTok{(}\StringTok{"The result is not significant at α ="}\NormalTok{, alpha, }
      \StringTok{"}\SpecialCharTok{\textbackslash{}n}\StringTok{(p{-}value ="}\NormalTok{, }\FunctionTok{round}\NormalTok{(chisq\_test}\SpecialCharTok{$}\NormalTok{p.value, }\DecValTok{4}\NormalTok{), }\StringTok{")."}\NormalTok{,}
      \StringTok{"}\SpecialCharTok{\textbackslash{}n}\StringTok{There is no evidence of a relationship between gender and performance.}\SpecialCharTok{\textbackslash{}n}\StringTok{"}\NormalTok{)}
\NormalTok{\}}
\end{Highlighting}
\end{Shaded}

\begin{verbatim}
## The result is significant at α = 0.05 
## (p-value = 0 ). 
## There is evidence of a relationship between gender and performance.
\end{verbatim}

\subsubsection{Visualization}\label{visualization}

\begin{Shaded}
\begin{Highlighting}[]
\CommentTok{\# Transform the data for plotting}
\NormalTok{plot\_data }\OtherTok{\textless{}{-}}\NormalTok{ study\_data }\SpecialCharTok{\%\textgreater{}\%}
  \FunctionTok{pivot\_longer}\NormalTok{(}\AttributeTok{cols =} \FunctionTok{c}\NormalTok{(Above\_Median, Below\_Median),}
               \AttributeTok{names\_to =} \StringTok{"Score\_Category"}\NormalTok{,}
               \AttributeTok{values\_to =} \StringTok{"Count"}\NormalTok{) }\SpecialCharTok{\%\textgreater{}\%}
  \FunctionTok{mutate}\NormalTok{(}\AttributeTok{Score\_Category =} \FunctionTok{factor}\NormalTok{(Score\_Category, }\AttributeTok{levels =} \FunctionTok{c}\NormalTok{(}\StringTok{"Above\_Median"}\NormalTok{, }\StringTok{"Below\_Median"}\NormalTok{)))}

\CommentTok{\# Plot}
\NormalTok{gender\_plot }\OtherTok{\textless{}{-}} \FunctionTok{ggbarplot}\NormalTok{(}
\NormalTok{  plot\_data, }\AttributeTok{x =} \StringTok{"Gender"}\NormalTok{, }\AttributeTok{y =} \StringTok{"Count"}\NormalTok{, }\AttributeTok{fill =} \StringTok{"Score\_Category"}\NormalTok{,}
  \AttributeTok{color =} \StringTok{"white"}\NormalTok{, }\AttributeTok{position =} \FunctionTok{position\_dodge}\NormalTok{(),}
  \AttributeTok{xlab =} \StringTok{"Gender"}\NormalTok{, }\AttributeTok{ylab =} \StringTok{"Count"}\NormalTok{,}
  \AttributeTok{legend.title =} \StringTok{"Score Category"}\NormalTok{,}
  \AttributeTok{title =} \StringTok{"Performance by Gender in Eye{-}Hand Coordination Test"}
\NormalTok{) }\SpecialCharTok{+}
  \FunctionTok{theme\_minimal}\NormalTok{()}
\NormalTok{gender\_plot}
\end{Highlighting}
\end{Shaded}

\includegraphics{Assigment2_files/figure-latex/unnamed-chunk-3-1.pdf}

\subsection{No-2(a) Doctor Radical, a math instructor at Logarithm
University, has three classes in advanced calculus. There are five
students in each class. The instructor uses a programmed textbook in
Class 1, a conventional textbook in Class 2, and his own printed notes
in Class 3. At the end of the semester, in order to determine if the
type of instruction employed influences student performance, Dr.~Radical
has another math instructor, Dr.~Root, rank the 15 students in the three
classes with respect to math ability. The rankings of the students in
the three classes follow: Class 1: 9.5, 14.5, 12.5, 14.5, 12.5; Class 2:
6, 9.5, 3, 9.5, 3; and Class 3: 1, 9.5, 6, 3, 6 (assume the lower the
rank, the better the student). At α = 0.05, is there a difference three
classes in advanced
calculus.?}\label{no-2a-doctor-radical-a-math-instructor-at-logarithm-university-has-three-classes-in-advanced-calculus.-there-are-five-students-in-each-class.-the-instructor-uses-a-programmed-textbook-in-class-1-a-conventional-textbook-in-class-2-and-his-own-printed-notes-in-class-3.-at-the-end-of-the-semester-in-order-to-determine-if-the-type-of-instruction-employed-influences-student-performance-dr.-radical-has-another-math-instructor-dr.-root-rank-the-15-students-in-the-three-classes-with-respect-to-math-ability.-the-rankings-of-the-students-in-the-three-classes-follow-class-1-9.5-14.5-12.5-14.5-12.5-class-2-6-9.5-3-9.5-3-and-class-3-1-9.5-6-3-6-assume-the-lower-the-rank-the-better-the-student.-at-ux3b1-0.05-is-there-a-difference-three-classes-in-advanced-calculus.}

\subsubsection{To input Dataset}\label{to-input-dataset}

The rankings of the students in the three classes are as follows: Class
1 (Programmed Textbook): 9.5, 14.5, 12.5, 14.5, 12.5 Class 2
(Conventional Textbook): 6, 9.5, 3, 9.5, 3 Class 3 (Printed Notes): 1,
9.5, 6, 3, 6

\begin{Shaded}
\begin{Highlighting}[]
\NormalTok{data }\OtherTok{\textless{}{-}} \FunctionTok{tibble}\NormalTok{(}
  \AttributeTok{Class =} \FunctionTok{rep}\NormalTok{(}\FunctionTok{c}\NormalTok{(}\StringTok{"Class 1"}\NormalTok{, }\StringTok{"Class 2"}\NormalTok{, }\StringTok{"Class 3"}\NormalTok{), }\AttributeTok{each =} \DecValTok{5}\NormalTok{),}
  \AttributeTok{Rank =} \FunctionTok{c}\NormalTok{(}
    \FloatTok{9.5}\NormalTok{, }\FloatTok{14.5}\NormalTok{, }\FloatTok{12.5}\NormalTok{, }\FloatTok{14.5}\NormalTok{, }\FloatTok{12.5}\NormalTok{,  }\CommentTok{\# Class 1}
    \DecValTok{6}\NormalTok{, }\FloatTok{9.5}\NormalTok{, }\DecValTok{3}\NormalTok{, }\FloatTok{9.5}\NormalTok{, }\DecValTok{3}\NormalTok{,             }\CommentTok{\# Class 2}
    \DecValTok{1}\NormalTok{, }\FloatTok{9.5}\NormalTok{, }\DecValTok{6}\NormalTok{, }\DecValTok{3}\NormalTok{, }\DecValTok{6}               \CommentTok{\# Class 3}
\NormalTok{  )}
\NormalTok{)}
\FunctionTok{print}\NormalTok{(data)}
\end{Highlighting}
\end{Shaded}

\begin{verbatim}
## # A tibble: 15 x 2
##    Class    Rank
##    <chr>   <dbl>
##  1 Class 1   9.5
##  2 Class 1  14.5
##  3 Class 1  12.5
##  4 Class 1  14.5
##  5 Class 1  12.5
##  6 Class 2   6  
##  7 Class 2   9.5
##  8 Class 2   3  
##  9 Class 2   9.5
## 10 Class 2   3  
## 11 Class 3   1  
## 12 Class 3   9.5
## 13 Class 3   6  
## 14 Class 3   3  
## 15 Class 3   6
\end{verbatim}

\subsubsection{Kruskal-Wallis Test}\label{kruskal-wallis-test}

\begin{Shaded}
\begin{Highlighting}[]
\NormalTok{kruskal\_result }\OtherTok{\textless{}{-}}\NormalTok{ data }\SpecialCharTok{\%\textgreater{}\%}
  \FunctionTok{kruskal\_test}\NormalTok{(Rank }\SpecialCharTok{\textasciitilde{}}\NormalTok{ Class)}
\CommentTok{\# Print the result}
\FunctionTok{print}\NormalTok{(kruskal\_result)}
\end{Highlighting}
\end{Shaded}

\begin{verbatim}
## # A tibble: 1 x 6
##   .y.       n statistic    df      p method        
## * <chr> <int>     <dbl> <int>  <dbl> <chr>         
## 1 Rank     15      8.75     2 0.0126 Kruskal-Wallis
\end{verbatim}

\subsubsection{Check for significance}\label{check-for-significance}

\begin{Shaded}
\begin{Highlighting}[]
\NormalTok{alpha }\OtherTok{\textless{}{-}} \FloatTok{0.05}  \CommentTok{\# Significance level}
\ControlFlowTok{if}\NormalTok{ (kruskal\_result}\SpecialCharTok{$}\NormalTok{p }\SpecialCharTok{\textless{}}\NormalTok{ alpha) \{}
  \FunctionTok{cat}\NormalTok{(}\StringTok{"The result is significant at α ="}\NormalTok{, alpha, }
      \StringTok{"}\SpecialCharTok{\textbackslash{}n}\StringTok{(p{-}value ="}\NormalTok{, }\FunctionTok{round}\NormalTok{(kruskal\_result}\SpecialCharTok{$}\NormalTok{p, }\DecValTok{4}\NormalTok{), }\StringTok{")."}\NormalTok{,}
      \StringTok{"}\SpecialCharTok{\textbackslash{}n}\StringTok{There is evidence that at least one group differs significantly.}\SpecialCharTok{\textbackslash{}n}\StringTok{"}\NormalTok{)}
\NormalTok{\} }\ControlFlowTok{else}\NormalTok{ \{}
  \FunctionTok{cat}\NormalTok{(}\StringTok{"The result is not significant at α ="}\NormalTok{, alpha, }
      \StringTok{"}\SpecialCharTok{\textbackslash{}n}\StringTok{(p{-}value ="}\NormalTok{, }\FunctionTok{round}\NormalTok{(kruskal\_result}\SpecialCharTok{$}\NormalTok{p, }\DecValTok{4}\NormalTok{), }\StringTok{")."}\NormalTok{,}
      \StringTok{"}\SpecialCharTok{\textbackslash{}n}\StringTok{There is no evidence that the groups differ significantly.}\SpecialCharTok{\textbackslash{}n}\StringTok{"}\NormalTok{)}
\NormalTok{\}}
\end{Highlighting}
\end{Shaded}

\begin{verbatim}
## The result is significant at α = 0.05 
## (p-value = 0.0126 ). 
## There is evidence that at least one group differs significantly.
\end{verbatim}

\subsubsection{Pairwise Comparisons}\label{pairwise-comparisons}

If the Kruskal-Wallis test is significant, we will perform pairwise
comparisons using the Dunn test with Bonferroni correction.

\begin{Shaded}
\begin{Highlighting}[]
\NormalTok{pairwise\_result }\OtherTok{\textless{}{-}}\NormalTok{ data }\SpecialCharTok{\%\textgreater{}\%}
  \FunctionTok{dunn\_test}\NormalTok{(Rank }\SpecialCharTok{\textasciitilde{}}\NormalTok{ Class, }\AttributeTok{p.adjust.method =} \StringTok{"bonferroni"}\NormalTok{) }\SpecialCharTok{\%\textgreater{}\%}
  \FunctionTok{add\_xy\_position}\NormalTok{(}\AttributeTok{x =} \StringTok{"Class"}\NormalTok{, }\AttributeTok{step.increase =} \FloatTok{0.2}\NormalTok{) }\SpecialCharTok{\%\textgreater{}\%}
  \FunctionTok{mutate}\NormalTok{(}\AttributeTok{label =} \FunctionTok{paste0}\NormalTok{(}\StringTok{"p = "}\NormalTok{, }\FunctionTok{signif}\NormalTok{(p.adj, }\DecValTok{4}\NormalTok{))) }\CommentTok{\# Adds x, y positions for annotation}
\FunctionTok{print}\NormalTok{(pairwise\_result)}
\end{Highlighting}
\end{Shaded}

\begin{verbatim}
## # A tibble: 3 x 14
##   .y.   group1  group2     n1    n2 statistic       p  p.adj p.adj.signif
##   <chr> <chr>   <chr>   <int> <int>     <dbl>   <dbl>  <dbl> <chr>       
## 1 Rank  Class 1 Class 2     5     5    -2.34  0.0193  0.0578 ns          
## 2 Rank  Class 1 Class 3     5     5    -2.74  0.00621 0.0186 *           
## 3 Rank  Class 2 Class 3     5     5    -0.396 0.692   1      ns          
## # i 5 more variables: y.position <dbl>, groups <named list>, xmin <dbl>,
## #   xmax <dbl>, label <chr>
\end{verbatim}

\subsubsection{Visualization}\label{visualization-1}

We will create a boxplot to visualize the rank distributions across the
three classes.

\begin{Shaded}
\begin{Highlighting}[]
\FunctionTok{ggplot}\NormalTok{(data, }\FunctionTok{aes}\NormalTok{(}\AttributeTok{x =}\NormalTok{ Class, }\AttributeTok{y =}\NormalTok{ Rank, }\AttributeTok{fill =}\NormalTok{ Class)) }\SpecialCharTok{+}
  \FunctionTok{geom\_boxplot}\NormalTok{(}\AttributeTok{outlier.color =} \StringTok{"red"}\NormalTok{, }\AttributeTok{outlier.shape =} \DecValTok{8}\NormalTok{) }\SpecialCharTok{+} \CommentTok{\# Boxplot}
  \FunctionTok{stat\_summary}\NormalTok{(}\AttributeTok{fun =} \StringTok{"mean"}\NormalTok{, }\AttributeTok{geom =} \StringTok{"point"}\NormalTok{, }\AttributeTok{shape =} \DecValTok{4}\NormalTok{, }\AttributeTok{size =} \DecValTok{3}\NormalTok{, }\AttributeTok{color =} \StringTok{"blue"}\NormalTok{) }\SpecialCharTok{+} \CommentTok{\# Mean point}
  \FunctionTok{labs}\NormalTok{(}
    \AttributeTok{title =} \StringTok{"Rank Distribution by Class"}\NormalTok{,}
    \AttributeTok{subtitle =} \FunctionTok{paste}\NormalTok{(}\StringTok{"Kruskal{-}Wallis p ="}\NormalTok{, }\FunctionTok{round}\NormalTok{(kruskal\_result}\SpecialCharTok{$}\NormalTok{p, }\DecValTok{4}\NormalTok{)),}
    \AttributeTok{x =} \StringTok{"Class"}\NormalTok{,}
    \AttributeTok{y =} \StringTok{"Rank"}
\NormalTok{  ) }\SpecialCharTok{+}
  \FunctionTok{theme\_minimal}\NormalTok{() }\SpecialCharTok{+}
  \FunctionTok{theme}\NormalTok{(}\AttributeTok{legend.position =} \StringTok{"none"}\NormalTok{)}
\end{Highlighting}
\end{Shaded}

\includegraphics{Assigment2_files/figure-latex/2(a)Visualization-1.pdf}

\subsection{No-2(b) A quality control study is conducted on a machine
that pours milk into containers. The amount of milk (in liters)
dispensed by the machine into 21 consecutive containers follows: 1.90,
1.99, 2.00, 1.78, 1.77, 1.76, 1.98, 1.90, 1.65, 1.76, 2.01, 1.78, 1.99,
1,76, 1.94, 1.78, 1.67, 1.87, 1.91, 1.91, 1.89. Are the successive
increments and decrements in the amount of milk dispensed
random?}\label{no-2b-a-quality-control-study-is-conducted-on-a-machine-that-pours-milk-into-containers.-the-amount-of-milk-in-liters-dispensed-by-the-machine-into-21-consecutive-containers-follows-1.90-1.99-2.00-1.78-1.77-1.76-1.98-1.90-1.65-1.76-2.01-1.78-1.99-176-1.94-1.78-1.67-1.87-1.91-1.91-1.89.-are-the-successive-increments-and-decrements-in-the-amount-of-milk-dispensed-random}

\subsubsection{Introduction}\label{introduction}

In this study, we aim to determine whether the successive increments and
decrements in the amount of milk dispensed by a machine are random. The
dataset contains the amount of milk (in liters) dispensed into 21
consecutive containers.

\begin{Shaded}
\begin{Highlighting}[]
\NormalTok{data }\OtherTok{\textless{}{-}} \FunctionTok{tibble}\NormalTok{(}
  \AttributeTok{Container =} \DecValTok{1}\SpecialCharTok{:}\DecValTok{21}\NormalTok{,}
  \AttributeTok{Milk\_Amount =} \FunctionTok{c}\NormalTok{(}\FloatTok{1.90}\NormalTok{, }\FloatTok{1.99}\NormalTok{, }\FloatTok{2.00}\NormalTok{, }\FloatTok{1.78}\NormalTok{, }\FloatTok{1.77}\NormalTok{, }\FloatTok{1.76}\NormalTok{, }\FloatTok{1.98}\NormalTok{, }\FloatTok{1.90}\NormalTok{, }\FloatTok{1.65}\NormalTok{, }\FloatTok{1.76}\NormalTok{,}
                  \FloatTok{2.01}\NormalTok{, }\FloatTok{1.78}\NormalTok{, }\FloatTok{1.99}\NormalTok{, }\FloatTok{1.76}\NormalTok{, }\FloatTok{1.94}\NormalTok{, }\FloatTok{1.78}\NormalTok{, }\FloatTok{1.67}\NormalTok{, }\FloatTok{1.87}\NormalTok{, }\FloatTok{1.91}\NormalTok{, }\FloatTok{1.91}\NormalTok{, }\FloatTok{1.89}\NormalTok{)}
\NormalTok{)}
\FunctionTok{print}\NormalTok{(data)}
\end{Highlighting}
\end{Shaded}

\begin{verbatim}
## # A tibble: 21 x 2
##    Container Milk_Amount
##        <int>       <dbl>
##  1         1        1.9 
##  2         2        1.99
##  3         3        2   
##  4         4        1.78
##  5         5        1.77
##  6         6        1.76
##  7         7        1.98
##  8         8        1.9 
##  9         9        1.65
## 10        10        1.76
## # i 11 more rows
\end{verbatim}

\subsubsection{Runs Test for Randomness}\label{runs-test-for-randomness}

To test the randomness of successive increments and decrements in the
amount of milk dispensed, we first create a binary sequence that
represents whether each successive value increases or decreases compared
to the previous value.

\begin{Shaded}
\begin{Highlighting}[]
\NormalTok{data }\OtherTok{\textless{}{-}}\NormalTok{ data }\SpecialCharTok{\%\textgreater{}\%}
  \FunctionTok{mutate}\NormalTok{(}\AttributeTok{Change =} \FunctionTok{ifelse}\NormalTok{(Milk\_Amount }\SpecialCharTok{\textgreater{}} \FunctionTok{lag}\NormalTok{(Milk\_Amount), }\DecValTok{1}\NormalTok{, }\DecValTok{0}\NormalTok{)) }\SpecialCharTok{\%\textgreater{}\%}
  \FunctionTok{drop\_na}\NormalTok{()}
\CommentTok{\# Perform Runs Test}
\NormalTok{runs\_test\_result }\OtherTok{\textless{}{-}} \FunctionTok{runs.test}\NormalTok{(data}\SpecialCharTok{$}\NormalTok{Change)}
\CommentTok{\# Display the result}
\NormalTok{runs\_test\_result}
\end{Highlighting}
\end{Shaded}

\begin{verbatim}
## 
##  Runs Test
## 
## data:  data$Change
## statistic = NaN, runs = 1, n1 = 9, n2 = 0, n = 9, p-value = NA
## alternative hypothesis: nonrandomness
\end{verbatim}

\subsubsection{Visualization}\label{visualization-2}

We can visualize the changes in the amount of milk dispensed to better
understand the pattern of increments and decrements.

\begin{Shaded}
\begin{Highlighting}[]
\FunctionTok{ggplot}\NormalTok{(data, }\FunctionTok{aes}\NormalTok{(}\AttributeTok{x =}\NormalTok{ Container, }\AttributeTok{y =}\NormalTok{ Milk\_Amount)) }\SpecialCharTok{+}
  \FunctionTok{geom\_line}\NormalTok{(}\AttributeTok{color =} \StringTok{"blue"}\NormalTok{, }\AttributeTok{size =} \DecValTok{1}\NormalTok{) }\SpecialCharTok{+}
  \FunctionTok{geom\_point}\NormalTok{(}\FunctionTok{aes}\NormalTok{(}\AttributeTok{color =} \FunctionTok{factor}\NormalTok{(Change)), }\AttributeTok{size =} \DecValTok{3}\NormalTok{) }\SpecialCharTok{+}
  \FunctionTok{scale\_color\_manual}\NormalTok{(}
    \AttributeTok{values =} \FunctionTok{c}\NormalTok{(}\StringTok{"0"} \OtherTok{=} \StringTok{"red"}\NormalTok{, }\StringTok{"1"} \OtherTok{=} \StringTok{"green"}\NormalTok{),}
    \AttributeTok{labels =} \FunctionTok{c}\NormalTok{(}\StringTok{"Decrease"}\NormalTok{, }\StringTok{"Increase"}\NormalTok{),}
    \AttributeTok{name =} \StringTok{"Change"}
\NormalTok{  ) }\SpecialCharTok{+}
  \FunctionTok{labs}\NormalTok{(}
    \AttributeTok{title =} \StringTok{"Milk Dispensed by Container"}\NormalTok{,}
    \AttributeTok{x =} \StringTok{"Container Number"}\NormalTok{,}
    \AttributeTok{y =} \StringTok{"Milk Amount (liters)"}
\NormalTok{  ) }\SpecialCharTok{+}
  \FunctionTok{theme\_minimal}\NormalTok{()}
\end{Highlighting}
\end{Shaded}

\begin{verbatim}
## Warning: Using `size` aesthetic for lines was deprecated in ggplot2 3.4.0.
## i Please use `linewidth` instead.
## This warning is displayed once every 8 hours.
## Call `lifecycle::last_lifecycle_warnings()` to see where this warning was
## generated.
\end{verbatim}

\includegraphics{Assigment2_files/figure-latex/2(b)visualization-1.pdf}

\subsection{No-3(a)A pediatrician speculates that the length of time an
infant is breast fed may be related to how often a child becomes ill.~In
order to answer the question, the pediatrician obtains the following two
scores for five three-year-old children: The number of months the child
was breast fed (which represents the X variable) and the number of times
the child was brought to the pediatrician's office during the current
year (which represents the Y variable). The scores for the five children
follow: Child 1 (20, 7); Child 2 (0, 0); Child 3 (1, 2); Child 4 (12,
5); Child 5 (3, 3). Do the data indicate that the length of time a child
is breast fed is related to the number of times a child is brought to
the
pediatrician?}\label{no-3aa-pediatrician-speculates-that-the-length-of-time-an-infant-is-breast-fed-may-be-related-to-how-often-a-child-becomes-ill.-in-order-to-answer-the-question-the-pediatrician-obtains-the-following-two-scores-for-five-three-year-old-children-the-number-of-months-the-child-was-breast-fed-which-represents-the-x-variable-and-the-number-of-times-the-child-was-brought-to-the-pediatricians-office-during-the-current-year-which-represents-the-y-variable.-the-scores-for-the-five-children-follow-child-1-20-7-child-2-0-0-child-3-1-2-child-4-12-5-child-5-3-3.-do-the-data-indicate-that-the-length-of-time-a-child-is-breast-fed-is-related-to-the-number-of-times-a-child-is-brought-to-the-pediatrician}

\begin{Shaded}
\begin{Highlighting}[]
\NormalTok{data }\OtherTok{\textless{}{-}} \FunctionTok{tibble}\NormalTok{(}
  \AttributeTok{Child =} \DecValTok{1}\SpecialCharTok{:}\DecValTok{5}\NormalTok{,}
  \AttributeTok{Months\_Breastfed =} \FunctionTok{c}\NormalTok{(}\DecValTok{20}\NormalTok{, }\DecValTok{0}\NormalTok{, }\DecValTok{1}\NormalTok{, }\DecValTok{12}\NormalTok{, }\DecValTok{3}\NormalTok{), }\CommentTok{\# X variable}
  \AttributeTok{Pediatrician\_Visits =} \FunctionTok{c}\NormalTok{(}\DecValTok{7}\NormalTok{, }\DecValTok{0}\NormalTok{, }\DecValTok{2}\NormalTok{, }\DecValTok{5}\NormalTok{, }\DecValTok{3}\NormalTok{)  }\CommentTok{\# Y variable}
\NormalTok{)}
\CommentTok{\# Print the dataset}
\NormalTok{data}
\end{Highlighting}
\end{Shaded}

\begin{verbatim}
## # A tibble: 5 x 3
##   Child Months_Breastfed Pediatrician_Visits
##   <int>            <dbl>               <dbl>
## 1     1               20                   7
## 2     2                0                   0
## 3     3                1                   2
## 4     4               12                   5
## 5     5                3                   3
\end{verbatim}

We calculate the Pearson correlation coefficient to determine the
relationship between the variables.

\begin{Shaded}
\begin{Highlighting}[]
\NormalTok{correlation }\OtherTok{\textless{}{-}} \FunctionTok{cor}\NormalTok{(data}\SpecialCharTok{$}\NormalTok{Months\_Breastfed, data}\SpecialCharTok{$}\NormalTok{Pediatrician\_Visits)}
\CommentTok{\# Display the correlation result}
\FunctionTok{cat}\NormalTok{(}\StringTok{"The Pearson correlation coefficient is:"}\NormalTok{, }\FunctionTok{round}\NormalTok{(correlation, }\DecValTok{3}\NormalTok{))}
\end{Highlighting}
\end{Shaded}

\begin{verbatim}
## The Pearson correlation coefficient is: 0.955
\end{verbatim}

To visualize the relationship, we create a scatter plot with a
regression line.

\begin{Shaded}
\begin{Highlighting}[]
\CommentTok{\# Create scatter plot}
\FunctionTok{ggplot}\NormalTok{(data, }\FunctionTok{aes}\NormalTok{(}\AttributeTok{x =}\NormalTok{ Months\_Breastfed, }\AttributeTok{y =}\NormalTok{ Pediatrician\_Visits)) }\SpecialCharTok{+}
  \FunctionTok{geom\_point}\NormalTok{(}\AttributeTok{size =} \DecValTok{4}\NormalTok{, }\AttributeTok{color =} \StringTok{"blue"}\NormalTok{) }\SpecialCharTok{+}
  \FunctionTok{geom\_smooth}\NormalTok{(}\AttributeTok{method =} \StringTok{"lm"}\NormalTok{, }\AttributeTok{se =} \ConstantTok{FALSE}\NormalTok{, }\AttributeTok{color =} \StringTok{"red"}\NormalTok{, }\AttributeTok{linetype =} \StringTok{"dashed"}\NormalTok{) }\SpecialCharTok{+}
  \FunctionTok{labs}\NormalTok{(}
    \AttributeTok{title =} \StringTok{"Breastfeeding Duration vs Pediatrician Visits"}\NormalTok{,}
    \AttributeTok{x =} \StringTok{"Months Breastfed"}\NormalTok{,}
    \AttributeTok{y =} \StringTok{"Number of Pediatrician Visits"}
\NormalTok{  ) }\SpecialCharTok{+}
  \FunctionTok{theme\_minimal}\NormalTok{()}
\end{Highlighting}
\end{Shaded}

\begin{verbatim}
## `geom_smooth()` using formula = 'y ~ x'
\end{verbatim}

\includegraphics{Assigment2_files/figure-latex/3(a)scatterplot-1.pdf}

\subsection{No-3(b) A study is conducted to determine whether there is a
correlation between handedness and eye-hand coordination. Five
right-handed and five left-handed subjects are administered a test of
eye-hand coordination. The test scores of the subjects follow (the
higher a subject's score, the better his or her eye-hand coordination):
Right-handers: 11, 1, 0, 2, 0; Left-handers: 11, 11, 5, 8, 4. Is there a
statistical relationship between handedness and eye hand
coordination?}\label{no-3b-a-study-is-conducted-to-determine-whether-there-is-a-correlation-between-handedness-and-eye-hand-coordination.-five-right-handed-and-five-left-handed-subjects-are-administered-a-test-of-eye-hand-coordination.-the-test-scores-of-the-subjects-follow-the-higher-a-subjects-score-the-better-his-or-her-eye-hand-coordination-right-handers-11-1-0-2-0-left-handers-11-11-5-8-4.-is-there-a-statistical-relationship-between-handedness-and-eye-hand-coordination}

\begin{Shaded}
\begin{Highlighting}[]
\NormalTok{data }\OtherTok{\textless{}{-}} \FunctionTok{tibble}\NormalTok{(}
  \AttributeTok{Handedness =} \FunctionTok{c}\NormalTok{(}\FunctionTok{rep}\NormalTok{(}\StringTok{"Right{-}handed"}\NormalTok{, }\DecValTok{5}\NormalTok{), }\FunctionTok{rep}\NormalTok{(}\StringTok{"Left{-}handed"}\NormalTok{, }\DecValTok{5}\NormalTok{)),}
  \AttributeTok{Score =} \FunctionTok{c}\NormalTok{(}\DecValTok{11}\NormalTok{, }\DecValTok{1}\NormalTok{, }\DecValTok{0}\NormalTok{, }\DecValTok{2}\NormalTok{, }\DecValTok{0}\NormalTok{, }\DecValTok{11}\NormalTok{, }\DecValTok{11}\NormalTok{, }\DecValTok{5}\NormalTok{, }\DecValTok{8}\NormalTok{, }\DecValTok{4}\NormalTok{)}
\NormalTok{)}
\CommentTok{\# Print the dataset}
\NormalTok{data}
\end{Highlighting}
\end{Shaded}

\begin{verbatim}
## # A tibble: 10 x 2
##    Handedness   Score
##    <chr>        <dbl>
##  1 Right-handed    11
##  2 Right-handed     1
##  3 Right-handed     0
##  4 Right-handed     2
##  5 Right-handed     0
##  6 Left-handed     11
##  7 Left-handed     11
##  8 Left-handed      5
##  9 Left-handed      8
## 10 Left-handed      4
\end{verbatim}

A boxplot is used to compare eye-hand coordination scores between
right-handed and left-handed subjects.

\begin{Shaded}
\begin{Highlighting}[]
\FunctionTok{ggplot}\NormalTok{(data, }\FunctionTok{aes}\NormalTok{(}\AttributeTok{x =}\NormalTok{ Handedness, }\AttributeTok{y =}\NormalTok{ Score, }\AttributeTok{fill =}\NormalTok{ Handedness)) }\SpecialCharTok{+}
  \FunctionTok{geom\_boxplot}\NormalTok{(}\AttributeTok{outlier.color =} \StringTok{"red"}\NormalTok{, }\AttributeTok{outlier.shape =} \DecValTok{8}\NormalTok{) }\SpecialCharTok{+}
  \FunctionTok{geom\_jitter}\NormalTok{(}\AttributeTok{width =} \FloatTok{0.2}\NormalTok{, }\AttributeTok{size =} \DecValTok{3}\NormalTok{, }\AttributeTok{alpha =} \FloatTok{0.7}\NormalTok{) }\SpecialCharTok{+}
  \FunctionTok{labs}\NormalTok{(}
    \AttributeTok{title =} \StringTok{"Eye{-}Hand Coordination by Handedness"}\NormalTok{,}
    \AttributeTok{x =} \StringTok{"Handedness"}\NormalTok{,}
    \AttributeTok{y =} \StringTok{"Eye{-}Hand Coordination Score"}
\NormalTok{  ) }\SpecialCharTok{+}
  \FunctionTok{theme\_minimal}\NormalTok{() }\SpecialCharTok{+}
  \FunctionTok{theme}\NormalTok{(}\AttributeTok{legend.position =} \StringTok{"none"}\NormalTok{)}
\end{Highlighting}
\end{Shaded}

\includegraphics{Assigment2_files/figure-latex/3(b)Visualizing the Data-1.pdf}
Null Hypothesis (H₀): There is no difference in eye-hand coordination
scores between right-handed and left-handed individuals. Alternative
Hypothesis(H₁): There is a difference in eye-hand coordination scores
between the groups.

\begin{Shaded}
\begin{Highlighting}[]
\CommentTok{\# Perform the Wilcoxon Rank{-}Sum Test}
\NormalTok{test\_result }\OtherTok{\textless{}{-}} \FunctionTok{wilcox.test}\NormalTok{(}
\NormalTok{  Score }\SpecialCharTok{\textasciitilde{}}\NormalTok{ Handedness,}
  \AttributeTok{data =}\NormalTok{ data,}
  \AttributeTok{exact =} \ConstantTok{FALSE}
\NormalTok{)}
\FunctionTok{print}\NormalTok{(test\_result)}
\end{Highlighting}
\end{Shaded}

\begin{verbatim}
## 
##  Wilcoxon rank sum test with continuity correction
## 
## data:  Score by Handedness
## W = 21, p-value = 0.08969
## alternative hypothesis: true location shift is not equal to 0
\end{verbatim}

\end{document}
